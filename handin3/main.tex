
\documentclass{article}
\usepackage{a4wide}
\usepackage[utf8]{inputenc}
\usepackage{amsmath}
\usepackage{mathtools}
\usepackage{amssymb}
\usepackage[english]{babel}
\usepackage{mdframed}
\usepackage{systeme,}
\usepackage{lipsum}
\usepackage{relsize}
\usepackage{graphicx}
\usepackage{caption}
\usepackage{tikz}
\usepackage{tikz-3dplot}
\usepackage{pgfplots}
\usepackage{harpoon}%
\usepackage{graphicx}
\usepackage{wrapfig}
\usepackage{subcaption}
\usepackage{a4wide}
\usepackage{comment}
\usepackage{authblk}
\usepackage{float}
\usepackage{listings}
\usepackage{xcolor}
\usepackage{amsmath}
\usepackage{chngcntr}
\usepackage{amsthm}
\usepackage{comment}
\usepackage{commath}
\usepackage{hyperref}%Might remove, adds link to each reference
\usepackage{url}
\newcommand{\w}{\omega}
\newcommand{\curl}[1]{\vec{\nabla}\times \vec{#1}}
\newcommand{\grad}{\vec{\nabla}}
\newcommand{\dive}[1]{\vec{\nabla}\cdot \vec{#1}}
%\newcommand{\crr}{\mathfrak{r}}
\usepackage{calligra}

\DeclareMathAlphabet{\mathcalligra}{T1}{calligra}{m}{n}
\DeclareFontShape{T1}{calligra}{m}{n}{<->s*[2.2]callig15}{}
\newcommand{\crr}{\mathcalligra{r}\,}
\newcommand{\boldscriptr}{\pmb{\mathcalligra{r}}\,}

\title{Handin 3, Fk 7048}
\author{Author : Andreas Evensen}
\date{Date: \today}
\definecolor{codegreen}{rgb}{0,0.6,0}
\definecolor{codegray}{rgb}{0.5,0.5,0.5}
\definecolor{codepurple}{rgb}{0.58,0,0.82}
\definecolor{backcolour}{rgb}{0.95,0.95,0.92}

\lstdefinestyle{mystyle}{
    backgroundcolor=\color{backcolour},   
    commentstyle=\color{codegreen},
    keywordstyle=\color{magenta},
    numberstyle=\tiny\color{codegray},
    stringstyle=\color{codepurple},
    basicstyle=\ttfamily\footnotesize,
    breakatwhitespace=false,         
    breaklines=true,                 
    captionpos=b,                    
    keepspaces=true,                 
    numbers=left,                    
    numbersep=5pt,                  
    showspaces=false,                
    showstringspaces=false,
    showtabs=false,                  
    tabsize=2
}

\lstset{style=mystyle}

\begin{document}

\maketitle
\section*{Warmup}

\subsection*{a)}
Suppose we have the following PDE:
\begin{align*}
    x\cdot u_x + y\cdot u_y = 2x,
\end{align*}then we can write down the following:
\begin{align*}
    \frac{dx}{x} = \frac{dy}{y} =\frac{du}{2x}.
\end{align*}Computing the following yields:
\begin{align*}
    \frac{dx}{x} &=\frac{dy}{y}\\
    \implies \ln(x) + \tilde{c}_1 &= \ln(y)\\
    x\cdot c_1 &= y\\
    \implies c_1 = \frac{y}{x}.
\end{align*}Similarly we have:
\begin{align*}
    \frac{dx}{x} &= \frac{du}{2x}\\
    \implies 2x + c_2 &= u\\
    \implies c_2 &= u - 2x.
\end{align*} We can now write down the following:
\begin{align*}
    G(c_1 ) &= c_2\\
    \implies G\Big(\frac{y}{x}\Big) &= u - 2x.
\end{align*}The boundary condition is given by:
\begin{align*}
    u(x, x^2) = x,
\end{align*}which implies the following:
\begin{align*}
    G\Big(\frac{y}{x}\Big)&=G\big(x\big) = u - 2x\\
    \implies u(x,y) &= -\Big(\frac{y}{x}\Big) + 2x.
\end{align*}

\section*{Problem 1}
\subsection*{1)}
Suppose we have the following PDE:
\begin{align*}
    \frac{\partial u}{\partial t} = \sigma \frac{\partial^2 u}{\partial x^2}
\end{align*}
\subsubsection*{a)}
We wish to determine the dimensions of $\sigma$ which we do by the following:
\begin{align*}
    \frac{1}{T} &= \sigma \frac{1}{L^2},\\
    [z] &= 1\triangleq xt^\alpha\sigma^\beta\\
    1&=LT^\alpha \Big(\frac{L^2}{T}\Big)^\beta,\\
    T^{\beta-\alpha}&= L^{2\beta + \alpha}\\
    \implies \alpha &= \beta =-\frac{1}{2}.
\end{align*}
\subsubsection*{b)}
We suppose we can decompose $u(x,t)$ as the following:
\begin{align*}
    u(x,t) =t^\delta\cdot f(z);\quad z \triangleq \frac{x}{\sqrt{t\sigma}}.
\end{align*} We thus compute the derivatives as follows:
\begin{align*}
    %\frac{\partial u}{\partial t} = \delta t^{\delta -1}f(z) + t^\delta \frac{\partial f(z)}{\partial t}\\
    \frac{\partial z}{\partial x} &= \frac{1}{\sqrt{t\sigma}}\\
    \frac{\partial z}{\partial t} &= \frac{-1}{2}\cdot\frac{x}{t^{3/2}\sqrt{\sigma}}\\
    \frac{\partial }{\partial t} &= \frac{\partial }{\partial z} \cdot \frac{\partial z}{\partial t}\\
    &= \frac{\partial }{\partial z} \cdot\Big(\frac{-1}{2}\cdot\frac{x}{t^{3/2}\sqrt{\sigma}}\Big)\\
    &=\frac{\partial}{\partial z}\cdot\Big(\frac{-1}{2}\frac{z}{t}\Big) = \frac{-z}{2t}\frac{\partial}{\partial z}, \\
    \frac{\partial^2 }{\partial x^2}&= \underbrace{\frac{\partial }{\partial z} \cdot \frac{\partial z}{\partial x}}_{\to u_x}\cdot \Big(\underbrace{\frac{\partial }{\partial z} \cdot \frac{\partial z}{\partial x}}_{\to u_x}\Big)\\
    &=\frac{\partial }{\partial z} \cdot \frac{1}{\sqrt{t\sigma}}\cdot \Big(\frac{\partial }{\partial z}\cdot \frac{1}{\sqrt{t\sigma}}\Big)\\
    &=\frac{1}{t\sigma}\frac{\partial ^2 }{\partial z^2}.
\end{align*}Applying these transformations yields the following:
\begin{align*}
    \frac{\partial}{\partial t}\Big(t^{\delta}\cdot f(z)\Big) &= \frac{1}{t} \frac{\partial^2}{\partial z^2}\Big(t^{\delta}f(z)\Big)\\
    \delta t^{\delta -1}f(z)- \frac{zt^{\delta -1}}{2}\frac{df}{dz} &= t^{\delta}\frac{d^2f}{dz^2}\\
    \implies \frac{d^2f}{dz^2} + \frac{z}{2}\frac{df}{dz} - \delta f &= 0.
\end{align*}

\subsubsection*{c)}
Setting $\delta = -\frac{1}{2}$ and solving with method of Frobenius yields the following:
\begin{align*}
    &\frac{d^2f}{dz^2} + \frac{z}{2}\frac{df}{dz} + \frac{1}{2}f= 0.\\
    &f(z) = \sum_{n = 0}^\infty a_n (z - z_0)^{n+r}\\
    &f'(z) = \sum_{n = 0}^\infty a_n (n+r) (z-z_0)^{n+r-1}\\
    &f''(z) = \sum_{n =0}^\infty a_n (n+r)(n+r-1)(z-z_0)^{n+r-2}\\
    \implies & \sum_{n = 0}^\infty a_n(n+r)(n+r-1)\cdot (z)^{n+r-2} + \frac{1}{2}\sum_{n = 0}^\infty a_n\cdot(n+r +1)\cdot(z)^{n+r} = 0.\\
    m= n - 2\implies&\sum_{m = -2}^\infty a_{m + 2}(m+r + 2)(m+r +1)\cdot (z)^{m+r} + \frac{1}{2}\sum_{n = 0}^\infty a_n\cdot(n+r + 1)\cdot(z)^{n+r} = 0,\\
\end{align*}
\begin{align*}
    \underbrace{a_{-2}(r)(r-1)z^{r-2}}_{ = 0} + \underbrace{a_{-1}(r)(r+1)z^{r-1}}_{=0} + &\sum_{n = 0}^\infty z^{n+r}\underbrace{\Big[a_{n+2}(n+r+1)(n+r+2) + \frac{a_n}{2}(n+r+1)\Big]}_{=0} = 0.\\
\end{align*}
There exists three cases for $r$: $r_1 = 0$, $r_2 = -1$ and $r_3 = 1$. We will discard the solution for $r_2 = -1$. The recursive relation from above is given by:
\begin{align*}
    \frac{a_{n+2}}{a_n} &= -\frac{n+r+1}{2(n+r+1)(n+r+2)}= -\frac{1}{2(n+r+2)}\\
    \text{$r = 0$ gives:}\quad  a_{n+2} &= -\frac{a_n}{2(n+2)}\\
    a_2 &= -\frac{a_0}{2(2)} = -\frac{a_0}{4}\\
    a_4 &= - \frac{a_2}{2(2 + 2)} = \frac{a_0}{4\cdot 8} = -\frac{a_0}{32}\\
    a_6 &= - \frac{a_4}{12} = \frac{a_0}{12 \cdot 32} = \frac{a_0}{384}\\
    a_{n+2} &= - \frac{a_n}{2(2+n)} = \frac{a_{n-2}}{2^2 (n + 2) (n)} = -\frac{a_{n-4}}{2^3(n+2)(n)(n-2)}\\
    &= \Big(-1\Big)^{n+1}\frac{a_0}{2^{n}(n+2)!!}\\
    a_{2n} &= (-1)^n \cdot \frac{\Big(\frac{1}{4}\Big)^n}{n!}a_0\\
    \implies f(z) &= a_0\exp\Big[-\frac{z^2}{4}\Big]
\end{align*}
\begin{comment}
\begin{align*}
    \text{$r = 1$ gives:}\quad  a_{n+2} &= -\frac{a_n}{2(n+3)}\\
    a_2 &= -\frac{a_0}{6}\\
    a_4 &= -\frac{a_2}{10} = \frac{a_0}{60}\\
    a_6 &= -\frac{a_4}{14} = -\frac{a_0}{840}\\
    a_{n+2} &= -1\cdot \frac{a_n}{2(n+3)} = 1\frac{a_{n-2}}{2^2(n+3)(n+1)} = \frac{a_{n-4}}{2^3(n+3)(n+1)(n-1)}\\
    a_{n+2} &= \Big(\frac{-1}{2}\Big)^{n+1}\frac{a_0}{2^{n-1}(n+3)!!},\\
    a_{2n} &= (-1)^{n+1} \frac{a_0}{(2n+1)!!}\\
\end{align*}
\end{comment}

\subsection*{2)}
Suppose the following equation:
\begin{align}
    u(x,y)\frac{\partial u}{\partial x} + v(x,y)\frac{\partial u}{\partial y} &= \lambda\frac{\partial^2 u}{\partial y ^2}\nonumber\\
    \frac{\partial u}{\partial x} + \frac{\partial v}{\partial y} &= 0.\label{eq: 2.2}
\end{align}
\subsection*{a)}
Given the following transformations:
\begin{align*}
    u &= \frac{\partial w}{\partial y},\\
    v &= -\frac{\partial w}{\partial x},
\end{align*}we wish to transform eq \eqref{eq: 2.2} into a PDE for $w(x,y)$. We compute the following:
\begin{align*}
    \frac{\partial u}{\partial x} &= \frac{\partial }{\partial x}\Big(\frac{\partial w}{\partial y}\Big)=\frac{\partial^2 w}{\partial x \partial y},\\
    \frac{\partial u}{\partial y} &= \frac{\partial }{\partial y}\Big(\frac{\partial w}{\partial y}\Big)=\frac{\partial^2 w}{\partial y^2}.
\end{align*}Plugging this into the original ODE, eq \eqref{eq: 2.2}, yields the following:
\begin{align*}
    \frac{\partial w}{\partial y}\cdot \frac{\partial^2 w}{\partial x\partial y} - \frac{\partial w}{\partial x} \cdot \frac{\partial^2w}{\partial y^2} = \lambda \frac{\partial^3 w}{\partial y^3}.
\end{align*}The above equation is now a third order PDE of $w(x,y)$. Moreover, we have by the boundary condition:
\begin{align*}
    \lim_{y\to \infty} \frac{\partial w}{\partial y} = U,
\end{align*}where $U$ is constant; thus $[w] = [U][y]$; since $U$ is a constant flow, or flux, we can impose that the dimensions of $w$ is that of flux per unit length.
\subsection*{b)}
We wish to find: $[x], [y], [\lambda]$ and $[w]$ and show the following:
\begin{align*}
    [y]=\Bigg[\sqrt{\frac{\lambda x}{U}}\Bigg],\quad [w]=\Big[\sqrt{U\lambda x}\Big].
\end{align*}Using what we found in the previous exercise we have the following:
\begin{align*}
    &[w] = [U][y]\\
    \implies& \Big[\frac{w}{y}\cdot\frac{w}{xy}\Big] = \Big[\frac{w}{x}\cdot\frac{w}{y^2}\Big] = \Big[\lambda \cdot \frac{w}{y^3}\Big]\\
    \implies& \Big[\frac{w^2}{xy^2}\Big] = \Big[\lambda \frac{w}{y^3}\Big]\\
    \implies & [y] =\Big[\sqrt{\frac{\lambda y x}{w}}\Big] =\Big[\sqrt{\frac{\lambda x}{U}}\Big]\\
    \implies & [w]^2 = \Big[\lambda\frac{w x}{y}\Big] \implies [w] = \Big[\sqrt{Ux\lambda}\Big]
\end{align*}
\subsection*{c)}
We look for solutions on the form:
\begin{align*}
    w(x,y) = \sqrt{U\lambda x}f(\eta); \quad \eta = y \sqrt{\frac{U}{\lambda x}}.
\end{align*} We want to find a third order differential equation in terms of $f(\eta)$, and we obtain this by the following computations:
\begin{align*}
    &\frac{\partial w}{\partial y}\cdot \frac{\partial^2 w}{\partial x\partial y} - \frac{\partial w}{\partial x} \cdot \frac{\partial^2w}{\partial y^2} = \lambda \frac{\partial^3 w}{\partial y^3}\\
    &\frac{\partial w}{\partial y} = \frac{\partial}{\partial y}\Big(\sqrt{U\lambda x}f(\eta)\Big) = \sqrt{U\lambda x}f'(\eta) \cdot \sqrt{\frac{U}{\lambda x}}\\
    &= Uf'(\eta)\\
    &\frac{\partial^2 w}{\partial y^2} = \frac{\partial }{\partial y}\Big(\frac{\partial w}{\partial y}\Big) = \frac{\partial}{\partial y}\Big(U f'(\eta)\Big)\\
    &= \sqrt{\frac{U^3}{\lambda x}}f''(\eta)\\
    &\frac{\partial^3 w}{\partial y^3} = \frac{\partial }{\partial y}\Big(\sqrt{\frac{U^3}{\lambda x}}f''(\eta)\Big)\\
    &= \frac{U^2}{\lambda x}f'''(\eta)\\
    &\frac{\partial w}{\partial x} = \frac{\partial}{\partial x}\Big(\sqrt{U\lambda x}f(\eta)\Big) = f(\eta)\frac{\partial}{\partial x}\Big(\sqrt{U \lambda x}\Big) + \sqrt{U\lambda x}f'(\eta)\frac{\partial}{\partial x}\Big(y\sqrt{\frac{U}{\lambda x}}\Big)\\
    &= f(\eta)\sqrt{\lambda U}\frac{1}{2\cdot x^{\frac{1}{2}}} + \sqrt{U\lambda x}f'(\eta)\sqrt{\frac{U}{\lambda x}}\frac{-y}{2\cdot x^{\frac{3}{2}}}\\
    &= \frac{\sqrt{U\lambda }}{2\cdot x^{\frac{1}{2}}}f(\eta) - \frac{yU}{2\cdot x}f''(\eta)\\
    &\frac{\partial^2w}{\partial x\partial y} = \frac{\partial}{\partial x}\Big(Uf'(\eta)\Big) = -Uf''(\eta)\sqrt{\frac{U}{\lambda}}\frac{1}{2\cdot x^{\frac{3}{2}}}
\end{align*}Using the properties of the partial derivatives we have the following PDE in terms of $w(x,y)$:
\begin{align*}
    &\frac{\partial w}{\partial y}\cdot \frac{\partial^2 w}{\partial x\partial y} - \frac{\partial w}{\partial x} \cdot \frac{\partial^2w}{\partial y^2} = \lambda \frac{\partial^3 w}{\partial y^3}\\
    &-\Big(Uf'(\eta)\cdot Uf''(\eta)\sqrt{\frac{U}{\lambda}}\frac{1}{2\cdot x^{\frac{3}{2}}}\Big)\\
    &-\Big(f(\eta)\sqrt{\lambda U}\frac{1}{2\cdot x^{\frac{1}{2}}} + \sqrt{U\lambda x}f'(\eta)\sqrt{\frac{U}{\lambda x}}\frac{-y}{2\cdot x^{\frac{3}{2}}}\Big)\cdot\Bigg(\sqrt{\frac{U^3}{\lambda x}}f''(\eta)\Bigg)=\frac{U^2}{x}f'''\\
    &\implies 2f'''(\eta) = - f'(\eta)\cdot f(\eta),
\end{align*}Which now is the so called Blasius equation.

\section*{Problem 2}
Suppose the following PDE:
\begin{align*}
    &y \frac{\partial u}{\partial x} + x\frac{\partial u}{\partial z} = u(x,y) - 1;\\
    &u(x, x = 2x) =x^2 + y + 1\quad \forall (x,y)\in\mathbb{R}^2
\end{align*}
Using $x = t$, $y = 2t$ and $u = (t+1)^2$ we get the following:
\begin{align*}
    x+ y  &= t + 2t = 3t \implies t = \frac{1}{3}(x+y) \\
    \implies u &= (t+1)^2\\
    u(x,y)&=\Big(\frac{1}{3}(x+y) + 1\Big)^2
\end{align*} This solution solves the PDE with the boundary conditions and due to uniqueness of solutions, this is the only solution to the PDE. 
\section*{Problem 3}
\subsection*{a)}
We wish to find the general solution to the differential equation:
\begin{align*}
    y''(x) + 2y'(x) + y(x) = 0.
\end{align*} We look for solutions on the form; $y(x) = e^{r x}$, which yields the following:
\begin{align*}
   e^{rx} \Big(r^2 + 2r + 1\Big) &= 0\\
    \implies r &= -1.
\end{align*}Since we have a double root (multiplicity) we have the following solution for the homogenous solution:
\begin{align*}
    y_h(x) = c_1 e^{-x} + c_2 x e^{-x} + c_3.
\end{align*}
\subsection*{b)}
\begin{align*}
    y''(x) + 2y'(x) + y(x) &= f(x);\\
    f(x)&= \begin{cases}
        \sin(x); x\in[0,2\pi]\\
        0; x\notin [0, 2\pi]
    \end{cases}
\end{align*}One notice that $f(0) = f(2\pi) = 0$ and thus we can divide the solution into two regions. Firstly, we will look for a particular solution in the region $x\in[0,2\pi]$.
\begin{align*}
    y_p(x) &= A\cos(x) + B\sin(x),\\
    y_p'(x) &= -A\sin(x) + B\cos(x)\\
    y_p''(x) &= -A\cos(x) - B\sin(x)\\
    \implies -A\cos(x) - B\sin(x) + &2\Big(-A\sin(x) + B\cos(x)\Big) + A\cos(x) + B\sin(x) = \sin(x)\\
    \implies -2A\sin(x) &= \sin(x)\quad A = -\frac{1}{2}.\\
    y_p(x) &= -\frac{1}{2}\cos(x)\\
\end{align*}Thus the particular solution in the region $x\in[0,2\pi]$ is given by:
\begin{align*}
    y = y_h + y_p = c_1 e^{-x} + c_2 x e^{-x} + c_3 - \frac{1}{2}\cos(x).
\end{align*} The boundary conditions yields that $c_3 = \frac{1}{2}$ which gives the final solution:
\begin{align*}
    y(x) = c_1 e^{-x} + c_2 x e^{-x} + \frac{1}{2} - \frac{1}{2}\cos(x).
\end{align*} The function $y(x)$ is $C^{(\infty)}$-smooth and thus is differentiable in every point on $x\in\mathbb{R}$.


\end{document}
    