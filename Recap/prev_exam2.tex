\section{Previous exam 2022}

\subsection*{Problem 1}
Solve the differential equation:
\begin{align*}
    x\frac{dy}{dx} + 2y = 10x^2; \quad y(1) = 3.
\end{align*}

\subsubsection*{Solution}

\subsection*{Problem 2}
Use a Green's function approach to solve the differential equation:
\begin{align*}
    y'' = 2x + 1; \quad y(0) = y'(L) = 0.
\end{align*}
\subsubsection*{Solution}

\subsection*{Problem 3}
Write down one equation that fulfills the following definitions. You do not need to solve equations that you write down:
\begin{enumerate}
    \item An ODE that is in Sturm-Liouville form. Explain why it is in Sturm Liouville form:
    \item A 2nd order PDE that could be solved using the Method of Characteristics:
    \item An ODE that is inhomogeneous:
    \item A contour integral that has a pole, but would evaluate to 0. Specify both the integral and the contour that you chose.
    \item A complex function that is continuous but nowhere differentiable.    
\end{enumerate}

\subsubsection*{Solution}

\subsection*{Problem 4}
Solve the differential equation using the Method of Frobenius:
\begin{align*}
    4xy'' + 2y' +y = 0.
\end{align*}Remember to write out full solutions for y1 and y2. These solutions may still be infinite sums.

\subsubsection*{Solution}

\subsection*{Problem 5}
Solve:
\begin{align*}
    \int_{1}^{\infty}\frac{\sin(x)}{x^2 + 3}\delta(x - 4)dx.
\end{align*}

\subsubsection*{Solution}

\subsection*{Problem 6}
Solve the integral:
\begin{align*}
    \int_{-\infty}^\infty \frac{1}{x^2 + 1}dx
\end{align*}

\subsubsection*{Solution}

\subsection*{Problem 7}
Calculate the Fourier transform of the following function:
\begin{align*}
    f(t) =\begin{cases}
        1; x\in[-T, 0]\\
        -1; x\in(0,T]\\
        0; \text{else}
    \end{cases}
\end{align*}

\subsubsection*{Solution}

\subsection*{Problem 8}
Two important relations for the Bessel function are:
\begin{align*}
    &J_{n-1}(x) + J_{n+1}(x) = \frac{2n}{x}J_n(x),\\
    &J_{n-1}(x) - J_{n+1}(x) = 2J_n'(x).
\end{align*}Use these to prove:
\begin{align*}
    \frac{d}{dx}\left[x^{-n}J_n(x)\right] = -x^{-n}J_{n + 1}(x)
\end{align*}

\subsubsection*{Solution}

\subsection*{Problem 9}
A particle is defined in a 3D rectangular box, of lengths $a \times b \times c$. The potential of the particle is 0 inside the box, but is infinite outside the box.
Using Schrödingers's equation in steady state:
\begin{align*}
    -\frac{\hbar^2}{2m}\nabla^2\psi + V\psi= E\psi,
\end{align*}what is the lowest energy state that this particle can have?

\subsubsection*{Solution}
