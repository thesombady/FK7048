\documentclass{article}
\usepackage{a4wide}
\usepackage[utf8]{inputenc}
\usepackage{mathtools}
\usepackage{amssymb}
\usepackage[english]{babel}
\usepackage{mdframed}
\usepackage{systeme,}
\usepackage{lipsum}
\usepackage{relsize}
\usepackage{caption}
\usepackage{tikz}
\usepackage{tikz-3dplot}
\usepackage{pgfplots}
\usepackage{harpoon}%
\usepackage{graphicx}
\usepackage{wrapfig}
\usepackage{subcaption}
\usepackage{authblk}
\usepackage{float}
\usepackage{listings}
\usepackage{xcolor}
\usepackage{amsmath}
\usepackage{chngcntr}
\usepackage{amsthm}
\usepackage{comment}
\usepackage{commath}
\usepackage{hyperref}%Might remove, adds link to each reference
\usepackage{url}
\newcommand{\w}{\omega}
\newcommand{\curl}[1]{\mathbf{\nabla}\times \mathbf{#1}}
\newcommand{\res}[2]{\text{Res}(#1, #2)}
\newcommand{\grad}{\mathbf{\nabla}}
\newcommand{\dive}[1]{\mathbf{\nabla}\cdot \mathfb{#1}}
%\newcommand{\crr}{\mathfrak{r}}
\usepackage{calligra}

\DeclareMathAlphabet{\mathcalligra}{T1}{calligra}{m}{n}
\DeclareFontShape{T1}{calligra}{m}{n}{<->s*[2.2]callig15}{}
\newcommand{\crr}{\mathcalligra{r}\,}
\newcommand{\boldscriptr}{\pmb{\mathcalligra{r}}\,}
\newcommand{\G}{\mathcal{G}}

\title{Handin 6}
\author{Author : Andreas Evensen}
\date{Date: \today}
\definecolor{codegreen}{rgb}{0,0.6,0}
\definecolor{codegray}{rgb}{0.5,0.5,0.5}
\definecolor{codepurple}{rgb}{0.58,0,0.82}
\definecolor{backcolour}{rgb}{0.95,0.95,0.92}

\lstdefinestyle{mystyle}{
    backgroundcolor=\color{backcolour},   
    commentstyle=\color{codegreen},
    keywordstyle=\color{magenta},
    numberstyle=\tiny\color{codegray},
    stringstyle=\color{codepurple},
    basicstyle=\ttfamily\footnotesize,
    breakatwhitespace=false,         
    breaklines=true,                 
    captionpos=b,                    
    keepspaces=true,                 
    numbers=left,                    
    numbersep=5pt,                  
    showspaces=false,                
    showstringspaces=false,
    showtabs=false,                  
    tabsize=2
}

\lstset{style=mystyle}

\begin{document}

\maketitle

\section{Warmup: Residue calculus}
\subsection*{a)}
We wish to compute the residue at $z = 0$ of the following function
\begin{align*}
    f(z) = z^{10}\cdot\exp\left[\frac{1}{z^2}\right]
\end{align*}
Firstly we note that we have a single pole at $z = 0$ for the expoential term, thus we can apply the following:
Laruent series expansion for $\exp\left[\frac{1}{z^2}\right]$:
\begin{align*}
    \exp\left[\frac{1}{z^2}\right] = \sum_{j = -\infty}^\infty a_j (-1)^j \frac{1}{j!(z)^{2j}}
\end{align*} Since the term only contains even powers of $z$, the coefficients for all odd powers is zero, and thus the residue at $0$ is $\res{f}{0}=0$.
\subsection*{b)}

We wish to compute the residue at $z = 0$, and $z=2$ of the following function
\begin{align*}
    f(z) &= \frac{\pi\cot(\pi z)}{z-2} = \frac{\pi}{z-2}\cdot\frac{\cos(\pi z)}{\sin(\pi z)}\\
    &=\frac{\pi\cos(\pi z)}{z-2}\cdot\frac{1}{z\pi - \frac{(z\pi)^3}{3!} + ...}\\
    &=\frac{\pi\cos(\pi z)}{z-2}\cdot\frac{1}{z\Big(\pi - \frac{z^2\pi^3}{3!} + ...\Big)}\\
\end{align*}Looking at the resiude at $z=0$ is done by the following, since $z=0$ is a single pole:
\begin{align*}
    \res{f}{0}=\lim_{z\to0}z\cdot f(z) = \lim_{z\to0}\frac{\pi\cos(z\pi)}{z-2}\cdot\frac{z}{z(\pi - \frac{z^2\pi^3}{3!} + ...)} = -\frac{1}{2}
\end{align*} For $z=2$ one does the following:
\begin{align*}
    f(z) = \frac{\pi\cot(\pi z)}{z-2}
\end{align*}There exists an simple pole at $z = 2$, and the residue is computed accordingly:
\begin{align*}
    \res{f}{2}&=\lim_{z\to2}\frac{\pi\cos(\pi z)\cdot(z-2)}{z-2}\cdot\frac{1}{z\Big(\pi - \frac{z^2\pi^3}{3!} + ...\Big)}\\
    &=\pi\cdot\frac{1}{2\Big(\pi - \frac{2^2\pi^3}{3!} + ...\Big)}\\
    &=\frac{1}{2}\cdot\frac{1}{1-\frac{(2\pi)^2}{3!} + ...}\\
    &=\frac{1}{2}\cdot\frac{1}{1 - \sum_{j = 2}^\infty \big((-1)^{j} \frac{(2\pi)^{2j}}{(2j + 1)!}\big)}\\
    &=\frac{1}{2}\cdot\frac{1}{\frac{2\pi^2}{3}}
\end{align*}Thus the residue is $\res{f}{2}=\frac{3}{4\pi^2}$.
\section{More Green's functions}
\subsection*{a)}
Suppose the following ODE:
\begin{align*}
    a\cdot y''(x) + b\cdot y'(x) + c\cdot y(x) = f(x); \quad (a,b,c) \in \mathbb{C}^3
\end{align*}
By defintion the Green's function $\mathcal{G}(x,s)$ is piece-wise $C^1$ and continous over the entire domain; thus the following holds:
\begin{align*}
    \mathcal{G}_1(s,s) = \mathcal{G}_2(s,s) 
\end{align*}Moreover, the discontinuity of the derivatives, the piece-wise $C^1$ property, is given by the following:
\begin{align*}
    &a\cdot\mathcal{G}''(x,s) + b\cdot\mathcal{G}'(x,s) + c\cdot\mathcal{G}(x,s) = \delta(x-s)\\
    &\lim_{\epsilon\to0}\left[a\int_{s-\epsilon}^{s + \epsilon}ds\left(\mathcal{G}''(x,s)\right) + b\underbrace{\int_{s-\epsilon}^{s + \epsilon}ds\left(\mathcal{G}'(x,s)\right)}_{=0} + c\underbrace{\int_{s - \epsilon}^{s + \epsilon} ds\left(\mathcal{G}(x,s)\right)}_{=0}\right] = 1\\
    &\lim_{\epsilon\to0}\mathcal{G}'(x,s)\Bigg|_{x = s - \epsilon}^{s + \epsilon} = \frac{1}{a}\\
    \implies& G'_2(s,s) - G'_1(s,s) = \frac{1}{a}
\end{align*}And thus the discontinuity condition is proven. Moreover, to make this a self-adjoint equation, one would impose the condition $a,b,c\in\mathbb{R}$ and such that $\frac{d}{dx}a = b$, i.e that one can write it on LS form.
If one does not write it on LS form, the differential operator is not Hermitian, and subsequentily, the Green's function is neither hermitian.

\subsection*{b)}
Suppose the following ODE:
\begin{align*}
    \begin{cases}
        y''(x) + 2\cdot y'(x) + y(x) = f(x)\\
        y(0) = 0\quad \&\quad y'(0) = 0
    \end{cases}.
\end{align*}We wish to compute $y(x)$ in terms of $f(x)$. The boundary conditions for $y(x)$ also apply for the Green's function, and thus one has:
\begin{align*}
    &\mathcal{G}''(x,s) + 2\cdot \mathcal{G}'(x,s) + \mathcal{G}(x,s) = \delta(x-s)\\
    \implies &\mathcal{G}(x,s) = 
    \begin{cases}
        A_1e^{-x} + B_1xe^{-x}; \quad x \in[x,s)\\
        A_2e^{-x} + B_2xe^{-x}; \quad x \in(s,L]\\
    \end{cases}
\end{align*}
From the boundary conditions one has the following:
\begin{align*}
    \mathcal{G}_1(0,s) &= A_1e^{0} + B_1\cdot(0)e^{0} = 0\implies A_1 =0\\
    \mathcal{G}_1'(0,s) &= -A_1e^{0} + B_1e^{0} - 0 B_1e^{0} = 0\implies B_1 = 0
\end{align*}
From the continuity condition one has the following:
\begin{align*}
    &\mathcal{G}_1(s,s) = \mathcal{G}_2(s,s)\\
    &0 = A_2e^{-s} + B_2s\cdot e^{-s}\\
    \implies &-A_2 = B_2s
\end{align*}
From the discontinuity condition one has the following:
\begin{align*}
    &\frac{d\mathcal{G}_2}{dx}\Bigg|_{x = s} - \frac{d\mathcal{G}_1}{dx}\Bigg|_{x = s} =1\\
    &-A_2e^{-s} + B_2e^{-s} - B_2se^{-s} = 1\\
    &B_2e^{-s} = 1\\
    \implies & B_2 = \frac{1}{e^{-s}}\\
    \implies & A_2 = -\frac{s}{e^{-s}}
\end{align*}
Thus the solution for $y(x)$ in terms of $f(x)$ is given by the following:
\begin{align*}
    y(x) &= \int_0^L\mathcal{G}(x,s)f(s)ds\\
    &= \underbrace{\int_0^x\mathcal{G}_1(x,s)f(s)ds}_{=0} + \int_x^L\mathcal{G}_2(x,s)f(s)ds\\
    &=\int_x^L \left[-\frac{s}{e^{-s}}e^{-x} + \frac{1}{e^{-s}}xe^{-x}\right]f(s)ds\\
\end{align*}
\subsection*{c)}
Suppose the following ODE, with the given boundary conditions:
\begin{align}
    \begin{cases}
        y''(x) + y'(x) - 2y(x) = f(x)\\
        y (0) = 1\\
        \lim_{x\to\infty}y(x) = 2
    \end{cases}\label{eq: task 2c ode}  
\end{align}
The differential operator $\mathcal{L}$ is defined and when acting on the Green's function $\mathcal{G}(x,s)$, one has the following:
\begin{align*}
    \mathcal{L} &= \frac{d^2}{dx^2} + \frac{d}{dx} - 2\\
    \mathcal{L}\mathcal{G}(x,s) &= \mathcal{G}''(x,s) + \mathcal{G}'(x,s) - \mathcal{G}(x,s) = \delta(x-s)
\end{align*}
For the cases of $x \neq s$ one has the following:
\begin{align*}
    &\G''(x,s) + \G'(x,s) - \G(x,s) = 0\\
    \text{Ansats: }&\G(x,s) = e^{\lambda x}\\
    \implies &e^{\lambda x}\underbrace{\left[\lambda^2 + \lambda -2\right]}_{=0} = 0\\
    &\lambda^2 + \lambda -2 = 0\\
    \implies \G(x,s)& = Ae^{-2x} + Be^{x}
\end{align*}This leads to the following defintion of $\G(x,s)$:
\begin{align*}
    \G(x,s) = \begin{cases}
        A_1e^{-2x} + B_1e^{x}; \quad x \in [0,s)\\
        A_2e^{-2x} + B_2e^{x}; \quad x \in (s,\infty)\\
    \end{cases}
\end{align*}One now states that the construction of a new function $u$ can be given as $u(x) = y(x) - g(x)$.
\begin{align*}
    u(x)'' + u(x)' -2u(x) &=Q(x)\\
    u''(x) + u'(x) - 2u(x) &= \underbrace{f(x) - g''(x) - g'(x) + 2g(x)}_{Q(x)}
\end{align*}Constructing a function $g$ such that is finite at $x = 0$ and vanishes as $x$ to infinity leads to:
\begin{align*}
    g(x) &= 2 - e^{-x}\\
    g'(x) &= +e^{-x}\\
    g''(x) &= - e^{-x}
\end{align*}The new boundary conditions are thus $0$ as $x$ tends towards infinity, and $0$ at $x = 0$.
\begin{align*}
    \G_1(0,s) &= A_1 + B_1 = 0 \implies A_1 = - B_1\\
    \G_2(\infty, s) &= B_2e^{\infty} = 0\implies B_2 = 0
\end{align*}The continuity condition then becomes:
\begin{align*}
    \G(s,s) &= \G_2(s,s)\\
    A_1e^{-2s} + B_1 e^{s} &= A_2e^{-2s}\\
    B_1e^{s} &= e^{-2s}(A_2 - A_1)
\end{align*}The discontinuity condition yields:
\begin{align*}
    &-2A_2e^{-2s} +2A_1e^{-2s} - B_1e^{s} = 1\\
    &2e^{s}B_1 - e^{s}B_1 = 1\\
    \implies&B_1 = \frac{1}{e^{s}}\\
    \implies&A_1 = -\frac{1}{e^{s}}\\
    \implies&A_2 = -\frac{1}{e^{s}}e^{-2s} + \frac{1}{e^s} e^{s} = -e^{-3s} + 1
\end{align*}
\begin{align*}
    y(x) &= u(x) + g(x)\\
    &= \left[\int_0^x ds\left(\G_1(x,s)(f(s) +2g(s) - g'(s) -g''(s))\right)\right] \\
    &+ \left[\int_x^L ds\left(\G_2(x,s)(f(s) +2g(s) - g'(s) -g''(s))\right)\right] +g(x)
\end{align*}
\subsection*{d)}
We now wish to compute the function $y(x)$ when $f = 1$, and show that it does not give the correct values:
\begin{align*}
  y(x) &= u(x) + g(x)\\
    &= \int_0^x ds\left(-e^{-s}e^{-2x} + e^{-s}e^{x}\right)\cdot\left(5-2e^{-s} -e^{-s} + e^{-s}\right)\\
    &+ \int_x^L ds\left((-e^{-3s} + 1)e^{-2x}\right)\cdot\left(5-2e^{-s} -e^{-s} + e^{-s} \right) +g(x)\\
    &= -e^{-2x}\int_0^x ds\left(e^{-s} \right)\cdot\left(5-2e^{-s} \right)\\
    &+ e^{x}\int_0^x ds\left( e^{-s}\right)\cdot\left(5-2e^{-s} \right)\\
    &+ e^{-2x}\int_x^L ds\left((-e^{-3s} + 1)\right)\cdot\left(5-2e^{-s} \right) +g(x)\\
    y(x) &= -e^{-2x}\Big(\dfrac{14 -  \mathrm{e}^{-3x}\cdot\left(15\mathrm{e}^{2x}-1\right)}{3} \Big) + e^{x}\Big(\dfrac{14 -  \mathrm{e}^{-3x}\cdot\left(15\mathrm{e}^{2x}-1\right)}{3} \Big)\\
    &+e^{-2x}\left(\dfrac{\left(150L+15\mathrm{e}^{-2L}+50\mathrm{e}^{-3L}-6e^{-5L}\right)}{30}-\dfrac{\left(150x+15\mathrm{e}^{-2x}+50\mathrm{e}^{-3x}-6e^{-5x}\right)}{30}\right)\\
    & + 2 - e^{-x}
\end{align*}
Computing $y(0)$ yields the following:
\begin{align*}
    y(0) = \underbrace{-\frac{\left(14 - (15-1)\right)}{3} + \frac{\left(14 - (15-1)\right)}{3}}_{=0} + 1 -\frac{41}{30} + R(L). 
\end{align*}The remainder: $R(L)$ is not zero at $x = 0$ and similarly, one can show that the boundary $x\to\infty$ is not equal to $2$.

\subsection*{e)}
In the case above, the function-space created is not complete, i.e the the basis of the Green's function can't be expressed as a sum over all of the basis-functions and their eigenvalues. In doing so, the solution is not unique,
and thus it's only a partial answer.

\section{Integral calculus}
\subsection*{a)}
The \underline{Estimation lemma}:\\\\\noindent
If a function $f\in\mathbb{C}$ is a contionous over a closed contour $\gamma$ and $|f(z)|\leq M$ for all $z\in\gamma$, then the following holds:
\begin{align*}
    \abs{\int_{\gamma}f(z)dz} \leq ML,
\end{align*}where $L$ is the arclength of the contour $\gamma$. The maximum $M$ is given by: $M = \max_{z\in\gamma}|f(z)|$.
\\\\\noindent
\underline{Jordan's lemma}:\\\\\noindent
A function $f\in\mathbb{C}$, integrated a semi-circular contour is bounded by the following:
\begin{align*}
    \abs{\int_\gamma f(z)dz} \leq \frac{\pi}{a}M,
\end{align*}if $f$ can be written as:
\begin{align*}
    f(z) = g(z)e^{iza}.
\end{align*}The bounded value $M$ is then defined as $M = \max_{\theta\in[0, \pi]}\abs{Re^{i\theta}}$.

\subsection*{b)}
\begin{align}
    I = \int_0^{2\pi}\frac{d\theta}{2 +\cos(\theta) + \sin(\theta)}\label{eq: 3b}
\end{align}
Letting $z = e^{i\theta}$ one would trace the unit circle, which is what the original integral required, eq \eqref{eq: 3b}. Thus one would have the following:
\begin{align*}
    &z = e^{i\theta}\quad dz = ie^{i\theta}d\theta\\
    \implies & -i\oint\frac{dz}{z\Big(2+\frac{e^{i\theta}+ e^{-i\theta}}{2} + \frac{e^{i\theta} - e^{-i\theta}}{2i}\Big)}\\
    &= -i\oint\frac{dz}{z\Big(2+\frac{1}{2}\big(z + z^{-1}\big)+\frac{1}{2i}\big(z-z^{-1}\big)\Big)}\\
    &= -i\oint\frac{dz}{\Big(2z+\frac{1}{2}\big(z^2 + 1\big)+\frac{1}{2i}\big(z^2-1\big)\Big)}\\
    &= -2i\oint\frac{dz}{\Big(4z+\big(z^2 + 1\big)-i\big(z^2-1\big)\Big)}\\
    & \frac{-2i}{1-i}\oint \frac{dz}{z^2 + \frac{4z}{1-i} + \frac{(1 + i)}{1-i}}\\
\end{align*}One now seeks the singularities in this expression and expands the expression:
\begin{comment}
\begin{align*}
    -2i\oint dz \left[\frac{1}{z - \underbrace{(1 - \frac{1}{\sqrt{2}} + i(-1 - \frac{1}{\sqrt{2}}))}_{z_1}}\right]\cdot\left[\frac{1}{z - \underbrace{(1 + \frac{1}{\sqrt{2}} + i(\frac{1}{\sqrt{2}} - 1))}_{z_0}}\right] 
\end{align*}
\end{comment}
\begin{align*}
    -\frac{2i}{1-i}\oint \frac{dz}{\Big(z -  \underbrace{(1 + \sqrt{2}) e^{-\frac{3\pi i }{4}}}_{z_1}\Big)\cdot\Big(z-\underbrace{(\sqrt{2} - 1)e^{-\frac{3\pi i}{4}}}_{z_0}\Big)}
\end{align*}

One notes that $z_1$ is outside the domain and it's a single pole, the residue is thus calculated:
\begin{align*} 
    \res{f}{z_0} &=\lim_{z\to z_0} (z - z_0)f(z)\\
    &=\lim_{z\to z_0}\frac{(z - z_0)}{(z - z_1)\cdot (z-z_0)}\\
    &=\lim_{z\to z_0}\frac{1}{(z - z_1)} = \frac{1}{\sqrt2(1+i)}\\
    \implies & 2\pi i \cdot \frac{-2i}{1-i}\frac{1}{\sqrt2(1+i)} = \sqrt{2}\pi
\end{align*}Thus the solution to the integral \eqref{eq: 3b} is $\sqrt{2}\pi$. 
\subsection*{c)}
\begin{align*}
    \int_{-\infty}^\infty e^{-(x + ia)^2}dx = \int_{-\infty}^\infty e^{-(x)^2}dx
\end{align*}
Assuming a contour form $(R,0)-(R,a)-(-R, a)-(-R,0)$ which closes itself, one can compute the closed contour integral in the following manner:
\begin{align*}
    \oint_C f(z)dz = \lim_{R\to\infty}\left[\int_{-R}^{R}e^{-x^2}dx + \underbrace{\int_0^{a}e^{-(R+iy)^2}dy}_{I_1} + \int_R^{-R}e^{-(x + ai)^2}dx+\underbrace{\int_{a}^{0}e^{-(-R+yi)^2}dy}_{I_2}\right].
\end{align*}The integrals $I_1$ and $I_2$ have the relationship $I_1 = -I_2$ and thus the closed contour integral becomes:
\begin{align*}
    \oint_C f(z)dz = \lim_{R\to\infty}\left[\int_{-R}^{R}e^{-x^2}dx + \int_R^{-R}e^{-(x + ai)^2}dx  \right]
\end{align*}There is no singularity and thus the closed integral becomes zero; one rearrange to the following:
\begin{align*}
    \lim_{R\to\infty} \int_{-R}^{R}e^{-x^2}dx &= -\lim_{R\to\infty}\int_R^{-R}e^{-(x + ai)^2}dx\\
    \underbrace{\lim_{R\to\infty} \int_{-R}^{R}e^{-x^2}dx}_{\sqrt{\pi}} &= \lim_{R\to\infty}\int_{-R}^{R}e^{-(x + ai)^2}dx\\
    \implies \sqrt{\pi} &= \int_{-\infty}^{\infty}e^{-(x + ai)^2}dx\\
\end{align*}

\subsection*{d)}
\begin{align*}
    J = \int_0^\infty\frac{\sin(x)}{x(x^2+a^2)}dx; \quad a\in[0,\infty)
\end{align*}
Then one can write the following:
\begin{align*}
    J = \frac{1}{2}\int_{-\infty}^\infty \frac{\sin(x)}{x(^2 + a^2)}dx
\end{align*}
Suppose the following:
\begin{align*}
    f(z) = \text{Im}\frac{e^{iz}}{z(z^2 + a^2)},
\end{align*}which has three singularities, $z_2 = 0$, $z_0 = ia$ and $z_1 = -ia$. Defining a semi-cricle on the first two quadrants removes the $z_1$ singularity, and having the closed contour being the farm from positive intifty 
to negative infity yields thus the following expression in terms of the residues:
\begin{align*}
    \res{f}{0} &= \lim_{z\to0} z \frac{e^{iz}}{z(z^2 + a^2)} = \frac{1}{a^2}\\
    \res{f}{z_0} &= \lim_{z\to z_0} (z - z_0)\frac{e^{iz}}{z(z - z_1)(z - z_0)} = \frac{e^{z_0}}{z_0(z_0 - z_1)} = -\frac{e^{-a}}{2a^2}
\end{align*}
% \res{f}{z_1} &= \lim_{z\to z_1} (z - z_1)\frac{\sin(z)}{z(z  - z_1)(z - z_0)} = \frac{\sin(z_1)}{z_1(z_1 - z_0)}\\
Thus we have the following:
\begin{align*}
    \oint_C dz &= 2\pi i\sum_{i}\res{f}{z_i} = \int_0^{\infty}\frac{\sin(x)}{x(x^2 + a^2)}dz + \int_{arc}dz + \int_{-\infty}^0\frac{\sin(x)}{x(x^2 +a^2)}dx\\
    &= 2\int_0^\infty \frac{\sin(x)}{x(x^2 + a^2)}dx\\
    \implies &\int_0^\infty\frac{\sin(x)}{x(x^2+a^2)}dx = \pi i\sum_i \res{f}{z_i} = \pi\Big(-\frac{e^{-a}}{2a^2} + \frac{1}{2a^2}\Big)\\
    &=\frac{\pi}{2a^2}\Big(\sinh(a) - \cosh(a) + 1\Big)
\end{align*}
When $a = 0$ the solution diverges, which is the same as the integral expression since it's then includes a essential singularity.

\end{document}
    