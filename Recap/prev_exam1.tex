\section{Previous exam 2021}

\subsection*{Problem 1}
Provide the formula for 4 of the 5 following equations and give an example of a physical scenario in which they may be employed:
\begin{enumerate}
    \item \textbf{Poisson's equation}:
    \item \textbf{Heat equation}:
    \item \textbf{Schrödinger's equation}:
    \item \textbf{Klein Gordon equation}:
    \item \textbf{Wave-equation}:
\end{enumerate}
\subsubsection*{Solution}
\begin{enumerate}
    \item \textbf{Poisson's equation}: $\nabla^2u = f$, when dealing with electrostatics.
    \item \textbf{Heat equation}: $\frac{\partial u}{\partial t} = \alpha \frac{\partial^2 u}{\partial x^2}$, when dealing with heat diffusion and heat transfer.
    \item \textbf{Schrödinger's equation}: $-\frac{\hbar^2}{2m}\nabla^2\psi + V\psi= E\psi$ when dealing with quantum mechanics; e.g. the hydrogen atom.
    \item \textbf{Klein Gordon equation}: $(\partial_\mu\partial^\mu + m^2)\phi = 0$, when dealing with relativistic quantum mechanics; e.g. the Klein Gordon field.
    \item \textbf{Wave-equation}: $\frac{\partial^t u}{\partial t^2} = c^2 \frac{\partial^2 u}{\partial x^2}$ when dealing with waves; e.g. sound waves.
\end{enumerate}

\subsection*{Problem 2}
Solve the differential equation:
\begin{align*}
    2xydx + (x^2 + 3y^2)dy = 0.
\end{align*}
\note You will only be able to find an implicit solution for $x$ and $y$ (but your solution should
not include their derivatives).

\subsubsection*{Solution}
This is a homogeneous ODE, so we can use the substitution $y = vx$ to get:
\begin{align*}
    &dy = vdx + xdv\\
    \implies& v(dx) + (x^2 + 3v^2x^2)(vdx + xdv) = 0\\
    \implies& vdx + x^2(1 + 3v^2)(vdx + xdv) = 0\\
\end{align*}

\subsection*{Problem 3}
Use contour integration to evaluate the integral:
\begin{align*}
    \int_{0}^\infty \frac{1}{x^3 + 1}dx.
\end{align*}

\subsubsection*{Solution}
Firstly one needs to find the residues of the integrand of:
\begin{align*}
    \oint_\Gamma \frac{1}{z^3 + 1}dz,
\end{align*}which is done by the following:
\begin{align*}
    z^3 - 1= (z+ 1)(z^2+z+1) = 0\implies z = -1, \frac{1}{2}\{1 + i\sqrt{3}, 1- i\sqrt{3}\}
\end{align*}which is a simple pole. The residue is then given by eq \eqref{eq:res1}:
\begin{align*}
    \res{f}{-1} &= \lim_{z\to-1}\frac{(z+1)}{(z+1)(z^2 + z + 1)} = 1.\\
    \res{f}{\frac{1}{2} + i\frac{\sqrt{3}}{2}}&=\lim_{z\to z_0}\frac{(z - z_0)}{(z+1)(z - z_0)(z -z_1)} = \frac{1}{z_0 + 1}\cdot\frac{1}{(z_0-z_1)}
\end{align*}One then defines a contour that goes from $0$ to $R$ on the real axis, and then a quarter-circle of radius $R$ in the upper half plane, where $R$ goes to infinity. Then:
\begin{align*}
    \oint_\Gamma\frac{1}{z^3 + 1}dz &= \int_{0}^R\frac{1}{x^3 + 1}dx + \underbrace{\int_{\Gamma_R}\frac{1}{z^3 + 1}dz}_{ = 0}\\
    \lim_{R\to\infty}\int_0^R\frac{1}{x^3 + 1}dx &= 2\pi i \frac{1}{\frac{1}{2} + i\frac{\sqrt{3}}{2} + 1}\cdot\frac{1}{(\frac{1}{2} + i\frac{\sqrt{3}}{2}-\frac{1}{2} - i\frac{\sqrt{3}}{2})}\\
    &= 2\pi i\frac{1}{\frac{1}{2} + \frac{i\sqrt{3}}{2}}\cdot\frac{1}{i\sqrt{3}}\\
    &=2\pi i\frac{1}{(\frac{i\sqrt{3}}{2} - \frac{3}{2})}
\end{align*}

\subsection*{Problem 4}
Write down one example that fulfills the definition for each of the following ODEs, you do not need to solve them:
\begin{enumerate}
    \item An ODE that is not linear, but is homogeneous:
    \item Has an order of 4:
    \item Is of degree 2:
    \item Is linear, but not homogeneous:
\end{enumerate}

\subsubsection*{Solution}
Remember that homogenuity is given by: $f(kx) = k^nf(x)$.
\begin{enumerate}
    \item \textbf{An ODE that is not linear, but is homogeneous:} $y' = -\frac{x}{y + 1}$
    \item \textbf{Has an order of 4:} $y^{(4)} + 1 = 0$.
    \item \textbf{Is of degree 2:} $y' = \sqrt{1 + x}$
    \item \textbf{Is linear, but not homogeneous:} $y' + 2y = 2e^{2x}$.
\end{enumerate}

\subsection*{Problem 5}
Find a general solution for this partial differential equation
\begin{align*}
    \frac{\partial u}{\partial t} - \lambda \frac{\partial^2u}{\partial x^2} = 0,
\end{align*}where $\lambda$ is taken to be a positive number and the solution fulfills boundary conditions $u(0, t) = u(L, t) = 0$.

\subsubsection*{Solution}
Using separation of variables, and moving one term to the other side:
\begin{align*}
    &T'(t)X(x) = \lambda T(t)X''(x)\\
    &\frac{T'(t)}{T(t)} = \lambda \frac{X''(x)}{X(x)} = \mu^2\\
    \implies T(t) &= Ae^{\mu^2 t}\\
    \implies X(x) &= a_1\sin\left(\frac{\mu}{\sqrt{\lambda}} x\right) + a_2\cos\left(\frac{\mu}{\sqrt{\lambda}} x\right)\\
\end{align*}The coefficient $a_2$ is zero from the $u(0,t) = 0$ boundary condition. The other boundary condition gives:
\begin{align*}
    X(L) = a_1\sin\left(\frac{\mu}{\sqrt{\lambda}} L\right) = 0\implies \frac{\mu}{\sqrt{\lambda}}L = n\pi\implies \mu_n = \frac{n\pi}{L}\sqrt{\lambda}
\end{align*}Thus the final solution is given by:
\begin{align*}
    u_n(x,t) = Ae^{\mu^2t}\sin\left(\mu_nx\right)
\end{align*}
\subsection*{Problem 6}
Take a break! You are more than halfway done. Don't put anything here (or potentially make a doodle in the space provided below) and you receive 5 points.

\subsubsection*{Solution}

\subsection*{Problem 7}
Provide all poles and calculate all residues for:
\begin{align*}
    f(z) = \frac{e^{iz}}{(z^2 + 1)(z+2)^2}.
\end{align*}It may be useful to remember:
\begin{align*}
    f^{(n)}(z_0) = \frac{n!}{2\pi i}\oint \frac{f(z)}{(z -z_0)^{n+1}}dz.
\end{align*}

\subsubsection*{Solution}
Rewriting the function as:
\begin{align*}
    f(z) = \frac{e^{iz}}{(z+ i)(z- i)(z+2)^2}.
\end{align*}From there one sees that there exists three poles, $z = i, -i, -2$, the first of the two poles are simple poles and last is a second order pole.
\begin{align*}
    \res{f}{i} &= \lim_{z\to i}\frac{(z-i)e^{iz}}{(z-i)(z+i)(z+2)^2} = \frac{e^{-1}}{2i(i + 2)^2}\\
    &=\frac{e^{-1}}{2i(3 + 4i)} = \frac{e^{-1}}{6i - 4}\\
    \res{f}{-i} &= \lim_{z\to -i}\frac{(z +i)e^{iz}}{(z-i)(z+i)(z+2)^2} = \frac{e}{-2i(-i + 2)^2}\\
    &=\frac{ie}{6- 8i}\\
    \res{f}{-2} &= \lim_{z\to -2}\frac{d}{dz}\left(\frac{(z-2)^2e^{iz}}{(z-i)(z+i)(z-2)^2}\right)\\
    &=\lim_{z\to -2}\frac{d}{dz}\left(\frac{e^{iz}}{(z+i)(z-i)}\right)\\
    &=\lim_{z\to-2} \frac{i e^{i z} (z^2 + 2 i z + 1)}{(z^2 + 1)^2}\\
    &=\frac{i e^{-2i} (4 -4 i  + 1)}{(4 + 1)^2} = \frac{ie^{-2i}(3-4i)}{25}
\end{align*}

\subsection*{Problem 8}
Calculate the Fourier transform of the triangle pulse:
\begin{align*}
    f(t) = \begin{cases}
        1-t; \quad t \in [0, 1]\\
        1 + t; \quad t \in [-1, 0)
    \end{cases}
\end{align*}

\subsubsection*{Solution}
Using the following definitions:
\begin{align*}
    \tilde{f}(\w) &=\int_{-\infty}^\infty f(t)e^{-i\w t}dt\\
    f(t) &= \frac{1}{2\pi}\int_{-\infty}^\infty \tilde{f}(\w)e^{i\w t}d\w
\end{align*}The transform is computed as:
\begin{align*}
    \tilde{f}(\w) &= \int_{-\infty}^\infty dt\left(f(t)e^{-i\w t}\right)\\
    &= \int_{-1}^0 (t+1)e^{-i\w t}dt + \int_{0}^1(1-t)e^{-i\w t}dt\\
    &= \int_{-1}^0 e^{-i\w t}dt + \int_{-1}^0 te^{-i\w t}dt + \int_{0}^1e^{-i\w t}dt - \int_{0}^1te^{-i\w t}dt\\
    &= \int_{-1}^1 e^{-i\w t}dt + \int_{-1}^0 te^{-i\w t}dt - \int_{0}^1te^{-i\w t}dt\\
    &= \Bigg[\frac{e^{-i\w t}}{-i\w}\Bigg]_{-1}^1 +\Bigg[\frac{te^{i\w t}}{-i\w}\Bigg]_{ -1}^1 - \int_{-1}^0 \frac{e^{-i\w t}}{-i\w}dt - \Bigg[\frac{te^{-i\w t}}{-i\w}\Bigg]_{0}^1 + \int_{0}^1\frac{e^{-i\w t}}{-i\w}dt\\
    &= \Bigg[\frac{e^{-i\w t}}{-i\w}\Bigg]_{-1}^1 +\Bigg[\frac{te^{i\w t}}{-i\w}\Bigg]_{ -1}^1 - \Bigg[\frac{e^{-i\w t}}{-\w^2}\Bigg]_{-1}^0 - \Bigg[\frac{te^{-i\w t}}{-i\w}\Bigg]_{0}^1 + \Bigg[\frac{e^{-i\w t}}{-\w^2}\Bigg]_{0}^1\\
    &= \frac{4}{\w}\sin(\w) + \frac{2}{\w^2}\left(\cos(\w) - 1\right)
\end{align*}

\subsection*{Problem 9}
The beta function is defined as:
\begin{align*}
    \beta(x,y) = \frac{\Gamma(x)\Gamma(y)}{\Gamma(x +y)}.
\end{align*}Compute $\beta(3,4)$.

\subsubsection*{Solution}
Since both are integers, one can use the factorial expression:
\begin{align*}
    \beta(3,4) = \frac{2!3!}{6!} = \frac{2}{6\cdot 5\cdot 4} = \frac{1}{6\cdot 5 \cdot 2} = \frac{1}{60}
\end{align*}

\subsection*{Problem 10}
Use the Method of Characteristics to solve:
\begin{align*}
    xu_x + yu_y = 2x
\end{align*}with the boundary conditions: $u(x,x^2) = x$.

\subsubsection*{Solution}
Using the Lagrange-Charpit equations, eq \eqref{eq: Lagrange-Charpit Equation}:
\begin{align*}
    \colorboxed{blue}{\frac{dx}{x} = \frac{dy}{y}} = \frac{du}{2x}.
\end{align*}Using the framed equations one gets:
\begin{align*}
    \frac{dy}{y} &= \frac{dx}{x} \\
    \implies \ln(y) &= \ln(x) + \tilde{c}_1\\
    \implies y &= c_1x\quad c_1 = \frac{y}{x}\\
\end{align*}Taking the any other combination of the equations, since the boundary condition contains $x$ one uses the $x$ derivative:
\begin{align*}
    \frac{dx}{x} &= \frac{du}{2x}\\
    2x + c_2 = u&\implies c_2 = u - 2x
\end{align*}Combining this gives:
\begin{align*}
    G(c_1) &= c_2\\
    G\left(\frac{y}{x}\right) &= u - 2x\\
\end{align*}Using the condition posed by $u(x, x^2) = x$ gives that the $y$-component can be written as $x^2$ and thus one has the following:
\begin{align*}
    G\left(\frac{y}{x}\right) &= G\left(\frac{x^2}{x}\right) = G(x) = u(x, x^2) - 2x = (x) - 2x = -x\\
\end{align*}So the function $G(x)$ only returns the negative of the argument. Thus the solution is given by:
\begin{align*}
    u(x,y) = -\left(\frac{y}{x}\right) + 2x
\end{align*}
\vspace*{4cm}
\begin{center}
    \color{red}
    \boxed{\color{black}\text{Previous exam 2021 is concluded here.}}
\end{center}